\chapter*{ВВЕДЕНИЕ}
\addcontentsline{toc}{chapter}{ВВЕДЕНИЕ}

Технология обнаружения объектов существует вообще везде. Наиболее очевидным приложением является программное обеспечение для разблокировки по распознаванию лиц на телефонах или системы камер видеонаблюдения в магазинах и складах.

С каждым годом её применение становится всё более актуальным, от обеспечения безопасности в общественных местах до автоматизации процессов в производстве \cite{intro1}. Эта технология позволяет распознавать и отслеживать объекты в реальном времени, что открывает новые возможности для инноваций и улучшения качества жизни. В условиях стремительного развития технологий и увеличения объёмов данных, обнаружение объектов становится неотъемлемой частью многих систем, включая автономные транспортные средства, системы видеонаблюдения, медицинские приложения и многое другое. Развитие этой технологии не только повышает уровень безопасности и эффективности, но и способствует созданию более умных и адаптивных решений для решения актуальных задач современности.

Целью работы является классификация методов подсчета количества человек на видео.
Для достижения поставленной цели необходимо решить следующие задачи:
\begin{enumerate}[leftmargin=1.6\parindent]
    \item провести анализ предметной области методов подсчета количества человек на видео;
    \item провести обзор существующих методов подсчета количества на видео;
    \item выделить критерии сравнения рассматриваемых методов и на этом основе классифицировать эти методы.
\end{enumerate}
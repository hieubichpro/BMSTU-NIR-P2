\chapter{Анализ предметной области}

\section{Задача подсчета количества человек на видео}

Подсчет количества человек на видео --- важная задача компьютерного зрения, которая используется в системах видеонаблюдения, анализа поведения, управлении потоками людей и других приложениях.
Она предполагает автоматическое определение количества людей в кадре на основе анализа изображений или видеопотока.

Основные этапы:
\begin{enumerate}[leftmargin=1.6\parindent]
    \item Обнаружение объекта --- определение местоположения людей на каждом кадре видео.
    Современные методы, такие как YOLO, Faster R-CNN или SSD, позволяют определять местоположение объектов на каждом кадре с высокой точностью и скоростью.
    \item Классификация объектов --- подтверждение, что обнаруженные объекты действительно являются людьми. Это достигается с помощью методов классификации, основанных на глубоких нейронных сетях, которые обучены различать людей от других объектов.
\end{enumerate}

Сложности задачи подсчета связаны с факторами, такими как перекрытие людей, изменения условий освещения и динамика сцены.
Тем не менее, развитие глубокого обучения и методов компьютерного зрения позволяет успешно справляться с этими вызовами.
Развитие современных методов детекции и глубокого обучения позволяет решать её с высокой точностью, что открывает широкие перспективы для применения в реальном времени.

Этот процесс представлен на рисунке~\ref{img:intro}.
\includeimage
	{intro}
	{f}
	{H}
	{1\textwidth}
	{Пример подсчета количества человек на видео}

\section{Существующие методы подсчета количества человек на видео}

\subsection{Алгоритм YOLO}
\textbf{YOLO} (You Only Look Once) — это один из самых популярных алгоритмов для обнаружения объектов в реальном времени. 
Он был разработан для решения задачи классификации объектов на изображениях с высокой скоростью и точностью. 
YOLO продемонстрировал свою способность быстро и эффективно идентифицировать, находить и распознавать объекты на изображениях. Вместо того, чтобы обрабатывать изображение как одно изображение, YOLO делит изображение на сетку из небольших ячеек. Каждая ячейка сетки считается отдельным «частичным изображением» и может принадлежать другому классу объектов. Эта идея позволяет YOLO обрабатывать несколько объектов на изображении за одно сканирование.

Такой подход помогает YOLO добиться впечатляющей производительности и высокой точности. Он решает такие проблемы, как перекрывающиеся объекты, частично затемненные объекты, объекты, появляющиеся на изображении в разных положениях и размерах, а также наличие фона изображения.

YOLO использует поля привязки для измерения и прогнозирования положения и взаимного расположения объектов, что позволяет учитывать множество различных форм объектов. В то же время YOLO также использует технику немаксимального подавления, чтобы исключить ненужные результаты и сохранить только самые важные объекты.

\textbf{YOLOv1} --- первоначальная модель YOLO рассматривала обнаружение объектов как проблему регрессии, что было значительным сдвигом от традиционного подхода к классификации.
\includeimage
	{yolov1}
	{f}
	{H}
	{1\textwidth}
	{Архитектура алгоритм YOLOv1}
В ходе тестирования YOLOv1 происходит умножение условных вероятностей классов на прогнозируемые значения уверенности для отдельных ячеек. Это позволяет получить оценки для каждого блока, которые соответствуют конкретному классу, согласно формуле \ref{eqq}
\begin{equation}
	\label{eqq}
	Pr(Class_{i} | Object) \times Pr(Object) \times IOU^{truth}_{pred} = Pr(Class_{i}) \times IOU^{truth}_{pred}
\end{equation}
Она использовала одну сверточную нейронную сеть (CNN) для обнаружения объектов на изображениях, разделяя изображение на сетку, делая несколько прогнозов на ячейку сетки, отфильтровывая прогнозы с низкой достоверностью, а затем удаляя перекрывающиеся блоки для получения окончательного вывода.

\textbf{YOLOv5} представлен в четырех основных версиях: маленькая (s), средняя (m), большая (l) и очень большая (x), каждая из которых обеспечивает постепенно возрастающий уровень точности. При этом каждая из версий требует разного времени на обучение: чем больше и точнее модель, тем дольше процесс тренировки.

На рисунке \ref{img:c-yolov5} представлено сравнение разных версий YOLOv5.
\includeimage
	{c-yolov5}
	{f}
	{H}
	{1\textwidth}
	{Сравнение разных версий YOLOv5}

Изображение проходит через входной слой (input) и передается в основную сеть (backbone) для извлечения признаков.
Основная сеть получает карты признаков различных размеров, которые затем объединяются через сеть слияния признаков (neck), чтобы в итоге сформировать три карты признаков: P3, P4 и P5 (в YOLOv5 размеры выражаются как 80×80, 40×40 и 20×20). Эти карты используются для обнаружения соответственно мелких, средних и крупных объектов на изображении.

После того как три карты признаков передаются в прогнозирующую голову (head), выполняются расчёт уровня уверенности (confidence) и регрессия ограничивающих рамок (bounding-box regression) для каждого пикселя карты признаков, используя заранее заданные якоря (prior anchors). Таким образом, получается многомерный массив (BBoxes), содержащий информацию о классе объекта, уровне уверенности класса, координатах рамки, ширине и высоте.

Затем, задавая соответствующие пороговые значения (confthreshold, objthreshold), фильтруется ненужная информация из массива. После этого выполняется процесс подавления немаксимумов (non-maximum suppression, NMS), чтобы получить финальную информацию об обнаруженных объектах.

Архитектура алгоритма YOLOv5 представлена на рисунке \ref{img:yolov5}
\includeimage
	{yolov5}
	{f}
	{H}
	{1\textwidth}
	{Архитектура YOLOv5}

\subsection{Алгоритм SSD}
\textbf{SSD} (Single Shot Detector) — это алгоритм для обнаружения объектов, который также ориентирован на высокую скорость и точность. Он был разработан для решения задачи локализации и классификации объектов на изображениях в реальном времени \cite{intro2}. Основная идея SSD заключается в том, что он использует однопроходную нейронную сеть для предсказания ограничивающих рамок и классов объектов, что позволяет ему эффективно обрабатывать изображения.
\includeimage
	{ssd}
	{f}
	{H}
	{1\textwidth}
	{Архитектура алгоритм SSD \cite{intro2}}
Основные характеристики алгоритма SSD можно описать следующими приведенными факторами:
\begin{enumerate}[leftmargin=1.6\parindent]
    \item Однопроходная архитектура: Как и YOLO, SSD выполняет обнаружение объектов за один проход через сеть, что значительно ускоряет процесс по сравнению с традиционными методами, требующими многократного анализа изображения.
    \item Многоуровневая структура: SSD использует несколько уровней (или слоев) для предсказания объектов, что позволяет ему обнаруживать объекты разных размеров. На каждом уровне сети генерируются ограничивающие рамки и вероятности классов, что улучшает точность обнаружения.
    \item Использование различных аспектов: Алгоритм SSD применяет различные соотношения сторон и масштабы для ограничивающих рамок, что позволяет ему более эффективно обнаруживать объекты с различными формами и размерами.
    \item Быстрая обработка: SSD обеспечивает высокую скорость обработки, что делает его подходящим для приложений, требующих реального времени, таких как видеонаблюдение, автономные транспортные средства и мобильные устройства.
    \item Обучение на больших наборах данных: SSD обучается на больших аннотированных наборах данных, что позволяет ему эффективно распознавать различные классы объектов и адаптироваться к различным условиям.
\end{enumerate}
\subsection{Модифицированные алгоритмы R-CNN}
R-CNN (Regions with Convolutional Neural Networks) представляет собой значительный шаг вперед в области обнаружения объектов, который был предложен Россом Бахи и его коллегами в 2014 году. Алгоритм использует двухступенчатый подход, который сочетает в себе методы селекции регионов и глубокое обучение. На первом этапе R-CNN генерирует набор предложений о регионах интереса (region proposals) с помощью алгоритма Selective Search, который выделяет потенциальные области, содержащие объекты. Эти регионы затем обрабатываются с использованием сверточной нейронной сети (CNN), что позволяет извлекать высокоуровневые признаки из каждого региона.

На втором этапе R-CNN применяет классификатор, обученный на извлеченных признаках, для определения класса объекта в каждом предложенном регионе. Для повышения точности алгоритм использует метод SVM (Support Vector Machine) для классификации и регрессию для уточнения границ объектов. Несмотря на свою эффективность, R-CNN имеет некоторые недостатки, такие как высокая вычислительная сложность и длительное время обработки, что связано с необходимостью обработки каждого региона отдельно. Эти ограничения стали основой для разработки более совершенных моделей, таких как Fast R-CNN и Faster R-CNN, которые стремятся улучшить скорость и точность обнаружения объектов, сохраняя при этом преимущества, заложенные в оригинальной архитектуре R-CNN.
\subsubsection{Алгоритм Fast R-CNN}
\textbf{Fast R-CNN} --- улучшенная версия R-CNN, разработанная Россом Гиршиком, которая устраняет основные недостатки R-CNN, такие как высокая вычислительная сложность и длительное время обработки \cite{fastrcnn}. Fast R-CNN использует слой ROI Pooling для преобразования предложений регионов (ROIs) в векторы фиксированной длины, что позволяет выполнять свёрточные операции один раз для всего изображения и делить результаты между всеми регионами. Это значительно ускоряет обработку и уменьшает затраты памяти.

Вместо трёх отдельных этапов (генерация регионов, извлечение признаков и классификация) Fast R-CNN объединяет их в единую архитектуру. В результате модель работает быстрее и эффективнее. Однако Fast R-CNN всё ещё зависит от медленного алгоритма Selective Search для генерации предложений регионов, что ограничивает её производительность. Этот недостаток исправляется в Faster R-CNN, где вводится собственная сеть для генерации регионов.

Архитектура алгоритма Fast R-CNN представлена на рисунке \ref{img:faster}
\includeimage
	{faster}
	{f}
	{H}
	{1\textwidth}
	{Архитектура Fast R-CNN \cite{fastrcnn}}

\subsubsection{Алгоритм Mask R-CNN}
\textbf{Mask R-CNN} — это нейронная сеть, предназначенная для задачи сегментации объектов (instance segmentation) в компьютерном зрении \cite{maskrcnn}. Она может выделять отдельные объекты на изображении или видео, предоставляя ограничивающие рамки (bounding boxes), классы объектов и их маски на уровне пикселей.

Архитектура алгоритма Mask R-CNN представлена на рисунке \ref{img:mask}
\includeimage
	{mask}
	{f}
	{H}
	{1\textwidth}
	{Архитектура Mask R-CNN}
Архитектура Mask R-CNN состоит из двух этапов. На первом этапе сеть генерирует предложения регионов (Region Proposal Network, RPN), которые могут содержать объекты. Для этого используется Feature Pyramid Network (FPN), которая обеспечивает извлечение признаков на разных масштабах. FPN включает в себя сверточную сеть (например, ResNet или VGG) для анализа изображения и механизм, сохраняющий информацию на разных уровнях разрешения.

На втором этапе сеть уточняет полученные регионы, определяет классы объектов, уточняет координаты ограничивающих рамок и создает маски объектов. В отличие от первого этапа, на этом этапе используется метод ROIAlign, который позволяет точно сопоставить области на карте признаков с исходным изображением, обеспечивая высокую точность результатов \cite{maskrcnn}.

Главное преимущество Mask R-CNN — способность работать с объектами на разных масштабах благодаря архитектуре FPN и использованию якорей (anchors). Это позволяет сохранять пространственные отношения между объектами на исходном изображении и их признаками на карте. Mask R-CNN активно используется в задачах медицинской диагностики, автономного вождения и видеоаналитики.


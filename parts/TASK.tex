\chapter{Сравнение методов подсчета количества человек на видео}

\section{Критерии сравнения методов подсчета количества человек на видео}

Для сравнения рассматриваемых методов будут использоваться следующие критерии~\cite{cmp}:
\begin{enumerate}[leftmargin=1.6\parindent]
	\item Точность (mAP).
	\item Скорость (FPS).
	\item Поддержка сегментации.
\end{enumerate}

Критерий <<mAP>> определяет, насколько хорошо метод распознает людей, избегая как ложных срабатываний, так и пропущенных объектов.
Критерий <<mAP>> вычисляется по следующей формуле~\ref{eq2}.
\begin{equation}
	\label{eq2}
	mAP = 1/N \sum_{i=1}^{N} AP_{i}
\end{equation}

, где N~---~ количество классов,
AP (average precision)~---~метрика, используемая для оценки точности работы детекторов объектов.

Критерий <<Скорость>> метода подсчета людей на видео определяется количеством кадров, обрабатываемых за секунду (FPS). Высокий FPS (30+) важен для задач реального времени, например, видеонаблюдения, а низкий FPS (<10) подходит для аналитики, где скорость менее критична. FPS является ключевым показателем эффективности метода.

Критерий <<Поддержка сегментации>> 
определяет возможность выделять маски объектов на уровне пикселей.
\section{Результаты сравнения}

В таблице~\ref{table1} представлено сравнение рассматриваемых методов~\cite{cmp}.
\clearpage
\begin{table}[!ht]
	\caption{\label{table1} Сравнение методов обнажурения объектов на основе сформированных критериев}
	\begin{tabularx}{\textwidth}{|X|c|c|c|}
		\hline
		Метод & mAP & FPS  & Поддержка сегментации \\ \hline
		YOLOv1 & 63 & 50 & Нет \\ \hline
		YOLOv5 & 68-73 & 60-120 & Нет \\ \hline
		SSD & 66 & 35-45 & Нет \\ \hline
		Fast R-CNN & 70-75 & 7-10 & Нет \\ \hline
		Mask R-CNN & 77 & 5-8 & Да \\ \hline
	\end{tabularx}
\end{table}

\section*{Вывод}

Из проведенного сравнения можно выделить несколько ключевых пунктов, которые подчеркивают основные преимущества и недостатки между рассматриваемыми методами:
\begin{enumerate}[leftmargin=1.6\parindent]
	\item YOLOv1 базовая модель, демонстрирующая ~63\% mAP на датасете COCO и высокую скорость обработки ~45-50 FPS. Она хорошо подходит для задач реального времени, однако её точность недостаточна для сложных сцен с мелкими объектами или плотными толпами. Модель не поддерживает сегментацию, что ограничивает её применение в задачах, требующих детального выделения объектов.
	\item YOLOv5 значительно превосходит YOLOv1 как по точности (mAP ~68-73\%), так и по скорости (FPS ~60-120, в зависимости от варианта модели). Она легко справляется с задачами реального времени и обеспечивает высокую производительность даже на сложных сценах. Однако, как и YOLOv1, она не поддерживает сегментацию, что может быть критично для некоторых задач.
        \item SSD достигает mAP ~66\% и поддерживает скорость ~35-45 FPS, что делает её хорошим выбором для задач, требующих баланса между точностью и производительностью. Модель особенно эффективна для мобильных устройств и систем видеонаблюдения. Однако она, как и YOLO, не поддерживает сегментацию, что ограничивает её применение в задачах с высоким уровнем детализации.
        \item Fast R-CNN демонстрирует высокую точность (mAP ~70-75\%) благодаря своей многостадийной архитектуре. Однако низкая скорость обработки (7-10 FPS) делает её непригодной для задач реального времени. Модель не поддерживает сегментацию, но остаётся полезной для оффлайн-анализа, где приоритетом является точность.
        \item Mask R-CNN обеспечивает самую высокую точность среди всех моделей (mAP ~77\%) и поддерживает сегментацию на уровне пикселей. Это делает её идеальной для сложных задач, таких как медицинская диагностика или анализ изображений с высокой детализацией. Однако её скорость (5-8 FPS) значительно уступает другим моделям, что ограничивает её использование в реальном времени.
\end{enumerate}

Таким образом, выбор модели зависит от требований задачи. Для реального времени предпочтительны YOLOv5 и SSD, для задач сегментации и высокой детализации — Mask R-CNN, а для оффлайн-анализа с упором на точность — Fast R-CNN.